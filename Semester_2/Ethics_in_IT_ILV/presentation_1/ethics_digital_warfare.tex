\documentclass[12pt]{beamer}
\usepackage{natbib}
\usepackage[backend=biber]{biblatex}
\addbibresource{sources.bib}
\usetheme{Boadilla}

\title{Ethik und digitale Kriegsführung}
\author{Philip Magnus}
\institute{FH Campus Wien}
\date{\today}

\begin{document}

\begin{frame}
    \titlepage
\end{frame}
\begin{frame}
    \frametitle{Inhaltsverzeichnis}
    \tableofcontents
\end{frame}

\section{Einführung}
\subsection{Was ist digitale Kriegsführung?}

\begin{frame}
    \frametitle{Was ist digitale Kriegsführung?}
    \begin{itemize}
        \item Definition: Einsatz digitaler Technologien in militärischen Konflikten
        \item Beispiele: Cyberangriffe, Drohnen, Informationskriegsführung 
    \end{itemize}
\end{frame}

\subsection{Begrifflichkeiten}

\begin{frame}
    \frametitle{Begrifflichkeiten}
    \begin{itemize}
        \item Cyberkrieg: kriegerische Auseinandersetzungen im und um den virtuellen Raum mit Mitteln der Informationstechnik (erstmals 1993 in einem Artikel von John Arquilla und David Ronfeldt) \cite{cyberkrieg}
        \item Autonome Waffensysteme (AWS): KI und Algorithmen sammeln Informationen und treffen Entscheidungen \cite{amnesty}
        \item Teilautonome Waffensysteme: sammelt und verarbeitet Daten ähnlich zu AWS, Mensch is in entscheidenden Phasen in Entscheidungen involviert
    \end{itemize}
\end{frame}

\section{Krieg und digitale Systeme}
\subsection{Auswirkungen digitaler Systeme}

\begin{frame}
    \frametitle{Cyberkrieg}
    \begin{itemize}
        \item Spionage: Eindringen in Computersysteme und Exfiltration von Infromationen \cite{cyberkrieg}
        \item Psychologische Kriegsführung: Social Engineering, Manipulation und Propaganda \cite{cyberkrieg}
        \item Sabotage: Einbringen von schadhafter Soft- und Hardware \cite{cyberkrieg}
        \item Bsp.: Stuxnet, DoS-Angirffe, etc.
    \end{itemize}
\end{frame}

\begin{frame}
    \frametitle{Autonome Waffensysteme (AWS)}
    \begin{itemize}
        \item Moderne Kriegsführung entwickelt sich zunehmend in Richtung vollständiger Automatisierung
        \item Faktor Mensch spielt für die Bedienung eine untergeordnete bis gar keine Rolle \cite{amnesty}
        \item Bsp.: Autonome Drohnen, Loitering Munition etc. \cite{amnesty}
    \end{itemize}
\end{frame}

\begin{frame}
    \frametitle{Teilautonome Waffensysteme}
    \begin{itemize}
        \item Mensch ist im Entscheidungsprozess des Waffensystems mit einbezogen
        \begin{itemize}
            \item Human-in-the-loop: Mensch trifft finale Entscheidung über Waffeneinsatz \cite{kas}
            \item Human-on-the-loop: System agiert selbstständig, Mensch überwacht und kann eingreifen \cite{kas}
        \end{itemize}
        \item Bsp.: "Fire and Forget"-Raketen, Patriot- o. Iron-Dome-Abwehrsysteme, etc.
    \end{itemize}
\end{frame}

\section{Ethik in digitaler Kriegsführung}
\subsection{Ethische Fragestellungen/Probleme}

\begin{frame}
    \frametitle{Ethische Fragestellungen I.}
    \begin{itemize}
        \item Mangelnde Menschliche Kontrolle
        \begin{itemize}
            \item unvorhersehbare Entscheidungen und Intransparenz\cite{d21}
            \item Menschen werden auf Daten reduziert \cite{d21}
            \item Wer ist noch Zivilist? \cite{netzpol}
        \end{itemize}
        \item Verantwortung u. Rechenschaft
        \begin{itemize}
            \item Wer trägt die Verantwortung für Fehlentscheidungen? \cite{d21}
            \item Zuordnung von Cyberangriffen ist schwierig \cite{rowe}
        \end{itemize}
        \item Verhältnismäßgikeit
        \begin{itemize}
            \item Wer trifft die Entscheidung über Verhältnismäßigkeit
            \item Wie wird Schadensbegrenzung und -bewertung erfasst? \cite{rowe}
        \end{itemize}
    \end{itemize}  
\end{frame}

\begin{frame}
    \frametitle{Ethische Fragestellungen II.}
    \begin{itemize}
        \item Regulierungsschwierigkeiten
        \begin{itemize}
            \item Schwere Nachverfolgung von Cyberwaffen
            \item Fehlende Transparenz \cite{dipert}
            \item Non-State Actors unterliegen keinen Regulierungen, z.B. Terror
        \end{itemize}
        \item Psychologische und gesellschaftliche Auswirkungen
        \begin{itemize}
            \item Manipulation der Gesellschaft durch Propaganda
            \item Entmenschlichung als Folge der digitalen Kriegsführung \cite{vaticannews}
        \end{itemize}  
    \end{itemize}
\end{frame}

\begin{frame}
    \frametitle{Aktuelle Entwicklungen}
    \begin{itemize}
        \item Internationale Initiativen
        \begin{itemize}
            \item Viele Staaten unterstützen Bestreben nach Regularien für AWS \cite{amnesty}
            \item UN-Generalsekretär drängt auf Abschluß von Verhandlungen zu Regulation bis 2026 \cite{amnesty}
        \end{itemize}
        \item Ethische Richtlinien für Einsatz und Entwicklung
        \begin{itemize}
            \item ethische Grundsätze bei Entwicklung von KI-Technologien \cite{vaticannews}
            \item Verhältnismäßigkeit muss auch bei digitalen Waffen erhalten bleiben \cite{dipert}
        \end{itemize}
    \end{itemize}    
     
\end{frame}

 
\begin{frame}[allowframebreaks]
    \printbibliography
\end{frame}

\end{document}