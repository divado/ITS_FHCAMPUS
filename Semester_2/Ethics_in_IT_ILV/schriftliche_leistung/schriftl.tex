\documentclass[12pt]{article}
\usepackage[utf8]{inputenc}
\usepackage[ngerman]{babel}

\title{Schriftliche Leistung zur Vorlesung ``Ethik in der IT''}
\author{Philip Magnus}
\date{\today}

\begin{document}

\maketitle

\section{Zwei ausgewählte Fragen mit Begründung}

\subsection*{Frage 1 - Kann eine solche Infografik-App den Trend zu verzerrten Aussagen verstärken oder ihm entgegenwirken?}

Ich glaube eine Infografik-App muss einige Anforderungen erfüllen, damit sie nicht zu einem Trend von verzerrten Aussagen beiträgt.\\
Es bedarf hier einiger Schutzmechanismen, wie z.B. Hinweisen zur Qualität der Daten, Quellennachweise und Kontext der Daten.\\
Durch eine leichte Zugänglichkeit, können auch Laien ohne die statistischen Kenntnisse hochwertig aussehende und überzeugende Grafiken erstellen.\\
Diese leichte Zugänglichkeit birgt aber auch gleichzeitig das größte Risiko der Anwendung, da hochwertig aussehende Grafiken eher bereitwillig von Menschen als Fakten akzeptiert werden, auch wenn die dahinterliegende Qualität der Daten eher gering ist.\\
Mit solchen irreführenden Grafiken können ``Fake News'' unterstützt und eventuell leichter verbreitet werden. Es bedarf hier wie bereits geschrieben Kontrollmechanismen, die Nutzer*innen dazu verpflichtet eine gewisse Datenqualität einzuhalten und Quellennachweise zu erbringen.\\

\subsection*{Frage 2 - Was bedeuten diese Anforderungen für Erstellerinnen von Datengrafiken? Wer trägt die Verantwortung dafür, dass eine Grafik richtig verstanden wird?}

Entwickler*innen wie Andrea und Maren haben genauso eine Verwantwortung wie die Nutzer*innen der App.\\
Bereits bei der Entwicklung eines Werkzeugs wie der Infografik-Appp müssen, die durch solch ein Werkzeug entstehenden, Missstände beachtet und diesen möglichst entgegen gewirkt werden.\\
Im Falle der Infografik-App könnte das z.B. durch eine verpflichtende Quellenangabe, Hinweise zum Kontext, die Angabe der Datenqualität, eine Erläuterung der Grafik und Hinweise auf potentiell missverständliche Darstellungen geschehen.\\
Sollte dies willentlich oder unwillentlich nicht geschehen, könnten die Entwickler*innen zur Verbreitung von Fehlinformationen beitragen.

\pagebreak

\section{Kritik an einer vorformulierten Frage}

\subsubsection*{Frage - Welchen praktischen Nutzen über schöne Bilder (und Marketing-Aktionen) hinaus hat das?}

Die Frage ist bereits im Tonfall negativ gestellt und impliziert, dass die App außer einer schönen Darstellung keinen Mehrwert bietet.\\
Es ist zwar berechtigt zu hinterfragen, welchen praktischen Nutzen die App bietet, dies liese sich aber besser und nicht wertend formulieren.\\
Anstelle einer provokanten Fragestellung wäre eine weniger wertende Formulierung für die Diskussion hilfreicher, z.B. ``Welche Anwendungsbereiche profitieren von einfach zugänglicher Datenvisualisierung, und wo sind Grenzen zu beachten?''\\
Durch die spezifischere und weniger wertende Fragestellung können sowohl positive als auch negaative Aspekte betrachtet werden, ohne dass die Frage bereits eine Richtung der Diskussion vorgibt.

\pagebreak

\section{Weiterführung des Fallbeispiels}

Wie jedes Jahr treffen sich die drei Freunde, im nächsten Jahr, wieder.\\
Aufgeret berichtet Andrea, dass ihre App mittlerweile von verschiedenen größeren Unternehmen und Organisationen genutzt wird, unter anderem von einer NGO um Klimadaten anschaulich darzustellen.\\
Allerdings muss sie ihren Freunden gegenüber auch eingestehen, dass die App von einem politisch extrem ausgerichteten YouTube-Kanal entdecht und verwendet wurde um mit ``wissenschaftlich belegten'' Daten seine Verschwörungstheorien zu untermauern.\\
Das Team rund um die App sieht sich nun mit Forderungen nach mehr Kontrolle konfrontiert. Nicht jedes Feedback ist aber konstruktiv, sie erhalten auch viele Hasskommentare.\\
Maren und Andrea haben deshalb bereits in einer Beta-Version ein sogenanntes ``Kontext-Overlay'' eingeführt und getestet, mit dem Quellen zu den erstellten Grafiken eingeblendet werden sollen. Das Menü soll auch noch um Hilfen zur Interpretation der Daten erweitert werden und es wird geprüft welche weiteren Maßnahmen vorgenommen werden können um die Verbreitung von Falschinformationen zu verhindern.\\
Das Vertrauen in die App ist trotzdem beschädigt und die Nutzer*innen sind gespalten, wie mit der App umgegangen werden soll.

\pagebreak

\section{Eigene ethische Frage zum Fallbeispiel}

\textbf{„Sollten Entwicklerinnen verpflichtet sein, technische Maßnahmen einzubauen, die irreführende Nutzung ihrer Software aktiv erschweren, auch auf Kosten der Nutzerfreundlichkeit?“}\\

Die Frage soll dazu anregen zu diskutieren, ob und was Entwickler*innen überhaupt tun können, um die Verbreitung von Falschinformationen zu verhindern und ob dies auf Kosten der Nutzer*innen geschehen darf.\\
Es soll die Frage aufgeworfen werden, inwieweit technische Neutralität überhaupt erreicht werden kann, oder ob Software auch automatisch immer eine politische Facette hat.\\
Der Stellenwert von Verantwortung durch Entwickler*innen kann und sollte hier kritisch hinterfragt werden.

\end{document}