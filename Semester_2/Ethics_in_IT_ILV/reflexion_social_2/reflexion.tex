\documentclass[12pt]{article}

\title{Fünf Fragen zur Hacker-Ethik}
\author{Philip Magnus}
\date{\today}

\begin{document}

\maketitle

\section{Reflexion zur vorlesung Social-Media II}


Die Im Rahmen der Vorlesung "Social-Media II" geführte Diskussion war intensiv und geprägt von provokanten Aussagen und Fragestellungen. Eine Diskussion darüber ob wir als Gesellschaft immer "dümmer" werden, war trotz gegenteilig erbrachten Beweisen geprägt von einer Anti-Haltung.
Das generelle Klima habe ich als sehr angespannt und teilweise unangenehm empfunden. Wir haben uns inhaltlich nicht wirklich aufeinander zubewegt und oft auch nicht wirklich Bezug auf die Standpunkte der anderen genommen.
Ich hatte das Gefühl es ging irgendwann nur noch darum, aus einer Trotzreaktion heraus, seine eigene Meinung zu vertreten und umbedingt recht zu behalten aber nicht darum voneinander etwas zu lernen.

\vspace{1em}

Im weitern Verlaufg der Vorlesung haben wir dann über dezentralisierte Social-Media Plattformen gesprochen und ob diese eine gute Alternative zu der aktuellen Landschaft der Social-Media Plattformen sind
Auch war es Thema, ob diese eine  Lösung sein können um bspw. Falschinformationen zu bekämpfen. Ich persönlich glaube, dass kollektives Eigentum an Social-Media Plattformen eine sehr gute Idee ist und einen großen Schritt im kampf gegen Fehlinformationen aber auch Zensur machen könnten.
Ein Modell ähnliuch zum öffentlich rechtlichen Rundfunk könnte hier vielleicht eine gute Lösung sein um diese Plattformen zu finazieren aber evtl. auch Menschen zur teilnahme zu bewegen.

\vspace{1em}

Im Rahmen meines IT-Security Studiums hätte ich mir jedoch einen technisch tieferen Einblick in die Funktionsweise dieser Plattformen gewünscht. Es wurde zwar angesprochen, war mir aber nicht technisch tief genug. Ich habe allerdings Verständnis dafür, dass dies im Rahmen einer Veranstatlung die sich mit Ethik in der IT beschäftigen soll vielleicht nicht der Fokus sein sollte.
Trotzdem empfinde ich es als wichtig ethische und technische Perspektiven miteinander zu verbinden und tiefergehend zu betrachten um ein umfassendes Verständnis für diese Technologien zu entwickeln.

\vspace{1em}

Nach der Pause wurde das Diskussionsklima etwas entspannter, ob das an der Pause und dem Durchatmen lag oder daran, dass viele Teilnehmer*innen evtl. keine Lust mehr hatten kann ich nicht wirklich sagen.
Jedenfalls wurde das Klima im Verlauf der Veranstaltung etwas entspannter.

Insgesammt nehme ich mir aus der Veranstaltung mit, wie wichtig es ist offen und respektvoll miteinander in Diskussionen umzugehen. Kontroverse Themen sind hier von Natur aus anfälliger dafür auszuarten und bedürfen vielleicht einer besonders gefühlvollen Moderation.
Die betrachteten teschnischen Themen zum status quo empfand ich als sehr interessant und werde sie mir auch nach der Veranstaltung noch einmal genauer anschauen.

\end{document}

