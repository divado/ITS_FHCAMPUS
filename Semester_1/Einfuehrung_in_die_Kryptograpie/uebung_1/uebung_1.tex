\documentclass[a4paper, 10pt]{article}
\usepackage{fullpage}
\usepackage{makecell}
\usepackage{amsmath}

\hyphenation{ent-spre-chen-der}

\title{Übung 1}
\author{Philip Magnus}
\date{\today}

\begin{document}
\maketitle
\section*{Aufgabe 1}

\begin{align*}
2640:2 \xrightarrow{} 1320:2 \xrightarrow{} 660:2 \xrightarrow{} 330:2 \xrightarrow{} 165:3 \xrightarrow{} 55:5 \xrightarrow{} 11\tag{a}
\\
2640 = 2*2*2*2*3*5*11
\\
\\
3829:7 \xrightarrow{} 547\tag{b}
\\
3829 = 7*547
\\
\\
6561:3 \xrightarrow{} 2187:3 \xrightarrow{} 729:3 \xrightarrow{} 243:3 \xrightarrow{} 81:3 \xrightarrow{} 27:3 \xrightarrow{} 9:3 \xrightarrow{} 3\tag{c}
\\
6561 = 3*3*3*3*3*3*3*3 = 3^{8}
\end{align*}

\section*{Aufgabe 2}

a)

\begin{center}
    \begin{tabular}{c|c|c|c} 
     \hline
     a & b & r & q \\ [0.5ex] 
     \hline\hline
     765 & 98 & 79 & 7 \\ 
     \hline
     98 & 79 & 19 & 1 \\
     \hline
     19 & 3 & 1 & 6 \\
     \hline
     3 & 1 & 0 & 3 \\
     \hline
    \end{tabular}
\end{center}

$\Rightarrow Teilerfremd$
\\
\\
b)

\begin{center}
    \begin{tabular}{c|c|c|c} 
     \hline
     a & b & r & q \\ [0.5ex] 
     \hline\hline
     234 & 18 & 0 & 13 \\ 
     \hline
    \end{tabular}
\end{center}


$\Rightarrow ggT: 18$
\\
\\
c)


\begin{center}
    \begin{tabular}{c|c|c|c} 
     \hline
     a & b & r & q \\ [0.5ex] 
     \hline\hline
     819 & 49 & 35 & 16 \\ 
     \hline
     49 & 35 & 14 & 1 \\
     \hline
     35 & 14 & 7 & 2 \\
     \hline
     14 & 7 & 0 & 2 \\
     \hline
    \end{tabular}
\end{center}


$\Rightarrow ggT: 7$
\\
\\
d)


\begin{center}
    \begin{tabular}{c|c|c|c} 
     \hline
     a & b & r & q \\ [0.5ex] 
     \hline\hline
     289 & 13 & 3 & 22 \\ 
     \hline
     13 & 3 & 1 & 4 \\
     \hline
     3 & 1 & 0 & 3 \\
     \hline
    \end{tabular}
\end{center}

$\Rightarrow Teilerfremd$

\section*{Aufgabe 3}

\begin{center} 
    \begin{tabular}{c|c|c} 
     \hline
     Ausdruck & Landau & Erklärung \\ [0.5ex] 
     \hline\hline
     $n^{\pi}+\pi^{n-1}$ & $O(\pi^{n})$ & $\pi > 1$ Term wächst exponentiell $\xrightarrow{} \infty$\\ 
     \hline
     $42n^{42}+(-1)^{24n}$ & $O(n^{42})$ & Term $42n^{42}$ dominiert, zweiter Term wechselt zwischen $+/-1$\\
     \hline
     $(n^{3}+3n^{2}-27)7$ & $O(n^{21})$ & $n^{21}+3n^{14}-27^{7}$, erster Term dominiert hier mit Wachstum\\
     \hline
     $(-2n)^{10}+0,3^{n+1}n$ & $O(n^{10})$ & \makecell{erster Term dominiert durch Wachstum\\ Koeffizient ist für Komplexität zu ignorieren,\\ zweiter Term mit Koeffzient $0,3$ sieht das Wachstum wie folgt aus $0,3 \xrightarrow{} 0$}\\
     \hline
     $e^{(i*\pi)*n}$ & $O(1)$ & \makecell{$e^{(i*\pi)*n} = (e^{(i*\pi)})^{n}$,\\ da $e^{(i*\pi)} = -1$ gilt $(e^{(i*\pi)})^{n} = (-1)^{n}$\\ Term wechselt also zwischen $+/-1$ Komplexität ist also Konstant}\\
     \hline
    \end{tabular}
\end{center}

\end{document}