\documentclass[12pt]{article}

\title{Fünf Fragen zur Hacker-Ethik}
\author{Philip Magnus}
\date{\today}

\begin{document}

\maketitle

\section{Welche Teile der Hackerethik haben dich überrascht? Warum?}

Besonders überrascht hat mich, wie sehr die Hacker von der Informationsfreiheit überzeugt sind und dafür auch eintreten. Den freien Fluss von Informationen als Grundrecht zu betrachten, ist eine sehr interessante 
und für mein Gefühl auch nobele Einstellung. Für seine Überzeugung auch Risiken einzugehen, ist nicht selbstversändlich.

\section{Wenn die Hackerethik um ein Prinzip erweitert werden könnte, welches sollte dies sein? Warum?}

Eine sinnvolle Erweiterung wäre meiner Meinung nach, dass Übernehmen von Verantwortung für das eigene Handeln. Die frühen Hacker waren sehr auf Funktionalität und Effizienz bedacht. Die Konsequenzen der eigenen Handlungen
sowie die Auswirkungen des technischen Fortschritts auf die Gesellschaft wurden quasi nicht reflektiert. Um die Hackerethik mehr in die Gesamtgesellschaft zu integrieren, wäre gerade die Reflexion des eigenen Handelns wichtig.

\section{Was macht eine Hacker*in aus?}

Hacker*innen sind Menschen, die sich intensiv mit Technik auseinandersetzen, aber gleichzeitig getrieben sind durch den Drang diese auch zu verbessern. Sie wollen verstehen wie die Systeme, die sie nutzen funktionieren,
teilen das Wissen hierüber aber auch gern mit anderen. Als Hacker*in sind auch sozialer Status nicht wichtig, sondern die Fähigkeit Probleme zu lösen und entsprechendes Wissen mit der Community zu teilen.

\section{Wellche Bedeutung hat die Freiheit in der Hackerethik?}

Wie schon in der ersten Frage erwähnt, hat die Freiheit einen sehr zentralen Stellenwert in der Hackerethik. Die Freiheit, Informationen zu teilen und zu verbreiten aber auch die persönliche Freiheut sich zu entfalten
und seinem Forscherdrang nachzugehen. Hierbei ist auch die Freiheit von bürokratischen Hürden und Autoritäten, welche den Austausch von Informationen einschränken könnte, wichtig.

\section{Ist die Hackerethik heute noch relevant? Warum? / Warum nicht?}

Die Hackerethik ist heute nach wie vor sehr releveant. Viele der Prinzipien, wie z.B. die Freiheit der Informationen, spielen auch heute noch eine zentrale Rolle in Hacker-Communities überall auf der Welt. Aber auch die
Möglichekeit der freien Entfaltung und Ablehnung von Diskriminierung sind nach wie vor sehr wichtig. Gerade in Zeiten, in denen immer mehr versucht wird den Fluss von Informationen einzuschränken oder zu kontrollieren, sind
Diskussionen und Gruppierungen, welche sich für Infoirmationsfreigheit einsetzen sehr relvant. Open-Source-Projekte oder Netzaktivist*innen tragen viel zu aktuellen Diskussionen rund umn digitale Themen bei und prägen
so auch hoffentlich die entsprechende Gesetzgebung und Gesellschaft.  

\end{document}