\documentclass[12pt]{article}
\usepackage[utf8]{inputenc}
\usepackage[ngerman]{babel}

\title{Schriftliche Leistung zur Vorlesung ``Ethik in der IT''}
\author{Philip Magnus}
\date{\today}

\begin{document}

\maketitle

\section{Zwei ausgewählte Fragen mit Begründung}

\subsection*{Frage 1 - Kann eine solche Infografik-App den Trend zu verzerrten Aussagen verstärken oder ihm entgegenwirken?}

Ich glaube eine Infografik-App muss einige Anforderungen erfüllen, damit sie nicht zu einem Trend von verzerrten Aussagen beiträgt.\\
Es bedarf hier einiger Schutzmechanismen, wie z.B. Hinweise zur Qualität der Daten, Quellennachweise und Kontext der Daten.\\
Durch eine leichte Zugänglichkeit, können auch Laien ohne die statistischen Kenntnisse hochwertig aussehende und überzeugende Grafiken erstellen.\\
Diese leichte Zugänglichkeit birgt aber auch gleichzeitig das größte Risiko der Anwendung, da hochwertig aussehende Grafiken eher bereitwillig von Menschen als Fakten akzeptiert werden, auch wenn die dahinterliegende Qualität der Daten eher gering ist.\\
Mit solchen irreführenden Grafiken können ``Fake News'' unterstützt und eventuell leichter verbreitet werden. Es bedarf hier wie bereits geschrieben Kontrollmechanismen, die Nutzer*innen dazu verpflichtet eine gewisse Datenqualität einzuhalten und Quellennachweise zu erbringen.\\

\subsection*{Frage 2 - Was bedeuten diese Anforderungen für Erstellerinnen von Datengrafiken? Wer trägt die Verantwortung dafür, dass eine Grafik richtig verstanden wird?}

Die Verantwortung liegt nicht allein bei den Nutzerinnen der App, sondern beginnt bereits bei den Entwicklerinnen wie Andrea und Maren. 
Wer ein Werkzeug zur Verfügung stellt, das weitreichende Wirkung entfalten kann, muss auch Verantwortung für die möglichen Missbräuche mitbedenken. 
In der Praxis bedeutet das: Transparenzfunktionen, Kontexthinweise, verpflichtende Erläuterungen zur Grafik und ggf. Warnsysteme bei potenziell missverständlichen Visualisierungen. 
Andernfalls riskieren die Entwickler*innen, zu unbeabsichtigten Mitverursachern von Desinformation zu werden.


\section{Kritik an einer vorformulierten Frage}

Frage: Welchen praktischen Nutzen über schöne Bilder (und Marketing-Aktionen) hinaus hat das?

Diese Frage impliziert bereits im Tonfall ein gewisses Misstrauen gegenüber dem Projekt und reduziert die App auf Oberflächlichkeit. Sie regt zwar berechtigt zur Reflexion über tatsächlichen Mehrwert an, bleibt aber in der Formulierung pauschal und wenig differenziert. Statt nur zu provozieren, wäre eine differenziertere Fragestellung hilfreicher, etwa: „Welche Anwendungsbereiche profitieren nachhaltig von niedrigschwellig zugänglicher Datenvisualisierung, und wo sind Grenzen zu beachten?“ So könnten sowohl positive als auch kritische Perspektiven angestoßen werden, ohne unterschwellig zu werten.


\section{Weiterführung des Fallbeispiels}

Ein Jahr später treffen sich die drei Freunde wieder. Andrea berichtet aufgeregt, dass ihre App inzwischen von einer NGO genutzt wird, um globale Klimadaten anschaulich darzustellen. Gleichzeitig aber hat ein politisch extrem orientierter YouTube-Kanal ebenfalls die App entdeckt – und nutzt sie, um aus den gleichen Daten scheinbar "wissenschaftlich belegte" Verschwörungstheorien zu untermauern. Das Projektteam wird mit Hasskommentaren, aber auch mit Forderungen nach mehr Kontrolle konfrontiert. In einer Beta-Version wird deshalb nun ein „Kontext-Overlay“ getestet, das automatisch Quellen einblendet und häufig genutzte Interpretationsfehler sichtbar macht – aber die Nutzer*innen sind gespalten.


\section{Eigene ethische Frage zum Fallbeispiel}

„Sollten Entwicklerinnen verpflichtet sein, technische Maßnahmen einzubauen, die irreführende Nutzung ihrer Software aktiv erschweren – selbst auf Kosten der Nutzerfreundlichkeit?“*

Diese Frage zielt darauf ab, ein Spannungsverhältnis offenzulegen: Zwischen der ethischen Verantwortung von Entwickler*innen und dem Prinzip der Gestaltungsfreiheit sowie Usability. Sie regt zur Diskussion darüber an, ob technische Neutralität überhaupt möglich ist – oder ob Software immer politisch und ethisch mitgestaltet. Sie lädt auch dazu ein, den Stellenwert von Verantwortung in der Softwareentwicklung kritisch zu hinterfragen.

\end{document}