\documentclass[12pt]{article}

\title{Fünf Fragen zur Hacker-Ethik}
\author{Philip Magnus}
\date{\today}

\begin{document}

\maketitle

\section{Reflexion zur vorlesung Social-Media I}

Die Vorlesung "Social-Media I" hat mich dazu gebracht meinen eigenen Umgang mit sozialen Medien einmal aktiv zu reflektieren und hinterfragen. Bereits mit den eingangs gestellten Fragen zu Umgang mit "Social Media" und die 
Einstellung zu gesellschaftlicher Entwicklung durch soziale Netzwerke haben bereits gezeigt, dass das Thema nicht so einfach zu behandeln ist, wie es vielleicht auf den ersten Blick scheint.

\vspace{1em}

Die Darstellungen und systematischen Aufschlüsselungen der verschriedenen Auswirkungen auf das wohlbefinden von Menschen haben mir noch einmal verdeutlicht, wie wichtig es ist, sich mit dem eigenen Konsum von sozialen Medien auseinanderzusetzen.
Ich persönlich habe aber das Gefühl, dass in der öffentlichen Diskussion oft zu sehr auf die negativen Aspekte fokussiert wird. Natürlich gibt es viele negative Aspekte, wie z.B. die Verbreitung von Falschinformationen oder die Gefahr der Sucht, aber soziale Medien bieten auch viele
positive Aspekte. Sie ermöglichen es Menschen, sich zu vernetzen, Informationen auszutauschen und Gemeinschaften zu bilden, die sonst vielleicht nicht möglich wären.

\vspace{1em}

Neben den persönlichen Konsequenzen, die soziale Medien auf das Wohlbefinden haben können, ist es auch wichtig, die gesellschaftlichen Auswirkungen zu betrachten. Das soziale Medien die Art und Weise, wie wir kommunizieren und interagieren, verändert haben, ist unbestritten.
Die neuen Herausforderungen, die sich daraus ergeben, wie z.B. die Verbreitung von Falschinformationen, müssen bekämpft werden, allerdings hat mir die Vorlesung auch noch einmal in Erinnerung gerufen, dass es auch zu positiven Veränderungen kommen kann, wie z.B. im "arabischen Frühling".

\vspace{1em}

Der Ausblich auf die weiteren Themen, besonders die Dezentralisierung von sozialen Medien hat mich dann noch einmal besonders neugiering gemacht. Ich persönlich empfinde die Idee von dezentralen und kollektiv besessenen sozialen Medien als einen wichtigen und richtigen Schritt
im Kamp gegen Falschinformationen und willkürliche Zensur.

\vspace{1em}

In der Diskussion über ein Tiktok--Verbot habe ich dann auch noch einmal die Chance bekommen, die Meinungen von anderen Studierenden zu hören und mich etwas aus meiner eigenen Blase zu entfernen. Andere Perspektiven sind gerade in diesem Kontext richtig un wichtig zu hören.

\vspace{1em}

Ob ich die interaktiven Elemente der Vorlesung, bspw. das beantworten von Fragestellungen durch Positionierung im Raum, als sinnvoll empfinde kann ich nicht wirklich beantworten. Mir ist die Intention dahinter bewusst, allerdings bräcuhte persönlich solche Elemente nicht um eine gute Diskussion zu starten.

\end{document}